
%% bare_conf.tex
%% V1.4b
%% 2015/08/26
%% by Michael Shell
%% See:
%% http://www.michaelshell.org/
%% for current contact information.
%%
%% This is a skeleton file demonstrating the use of IEEEtran.cls
%% (requires IEEEtran.cls version 1.8b or later) with an IEEE
%% conference paper.
%%
%% Support sites:
%% http://www.michaelshell.org/tex/ieeetran/
%% http://www.ctan.org/pkg/ieeetran
%% and
%% http://www.ieee.org/

%%*************************************************************************
%% Legal Notice:
%% This code is offered as-is without any warranty either expressed or
%% implied; without even the implied warranty of MERCHANTABILITY or
%% FITNESS FOR A PARTICULAR PURPOSE! 
%% User assumes all risk.
%% In no event shall the IEEE or any contributor to this code be liable for
%% any damages or losses, including, but not limited to, incidental,
%% consequential, or any other damages, resulting from the use or misuse
%% of any information contained here.
%%
%% All comments are the opinions of their respective authors and are not
%% necessarily endorsed by the IEEE.
%%
%% This work is distributed under the LaTeX Project Public License (LPPL)
%% ( http://www.latex-project.org/ ) version 1.3, and may be freely used,
%% distributed and modified. A copy of the LPPL, version 1.3, is included
%% in the base LaTeX documentation of all distributions of LaTeX released
%% 2003/12/01 or later.
%% Retain all contribution notices and credits.
%% ** Modified files should be clearly indicated as such, including  **
%% ** renaming them and changing author support contact information. **
%%*************************************************************************


% *** Authors should verify (and, if needed, correct) their LaTeX system  ***
% *** with the testflow diagnostic prior to trusting their LaTeX platform ***
% *** with production work. The IEEE's font choices and paper sizes can   ***
% *** trigger bugs that do not appear when using other class files.       ***                          ***
% The testflow support page is at:
% http://www.michaelshell.org/tex/testflow/



\documentclass[conference]{IEEEtran}
% Some Computer Society conferences also require the compsoc mode option,
% but others use the standard conference format.
%
% If IEEEtran.cls has not been installed into the LaTeX system files,
% manually specify the path to it like:
% \documentclass[conference]{../sty/IEEEtran}





% Some very useful LaTeX packages include:
% (uncomment the ones you want to load)


% *** MISC UTILITY PACKAGES ***
%
%\usepackage{ifpdf}
% Heiko Oberdiek's ifpdf.sty is very useful if you need conditional
% compilation based on whether the output is pdf or dvi.
% usage:
% \ifpdf
%   % pdf code
% \else
%   % dvi code
% \fi
% The latest version of ifpdf.sty can be obtained from:
% http://www.ctan.org/pkg/ifpdf
% Also, note that IEEEtran.cls V1.7 and later provides a builtin
% \ifCLASSINFOpdf conditional that works the same way.
% When switching from latex to pdflatex and vice-versa, the compiler may
% have to be run twice to clear warning/error messages.






% *** CITATION PACKAGES ***
%
%\usepackage{cite}
% cite.sty was written by Donald Arseneau
% V1.6 and later of IEEEtran pre-defines the format of the cite.sty package
% \cite{} output to follow that of the IEEE. Loading the cite package will
% result in citation numbers being automatically sorted and properly
% "compressed/ranged". e.g., [1], [9], [2], [7], [5], [6] without using
% cite.sty will become [1], [2], [5]--[7], [9] using cite.sty. cite.sty's
% \cite will automatically add leading space, if needed. Use cite.sty's
% noadjust option (cite.sty V3.8 and later) if you want to turn this off
% such as if a citation ever needs to be enclosed in parenthesis.
% cite.sty is already installed on most LaTeX systems. Be sure and use
% version 5.0 (2009-03-20) and later if using hyperref.sty.
% The latest version can be obtained at:
% http://www.ctan.org/pkg/cite
% The documentation is contained in the cite.sty file itself.






% *** GRAPHICS RELATED PACKAGES ***
%
\ifCLASSINFOpdf
  % \usepackage[pdftex]{graphicx}
  % declare the path(s) where your graphic files are
  % \graphicspath{{../pdf/}{../jpeg/}}
  % and their extensions so you won't have to specify these with
  % every instance of \includegraphics
  % \DeclareGraphicsExtensions{.pdf,.jpeg,.png}
\else
  % or other class option (dvipsone, dvipdf, if not using dvips). graphicx
  % will default to the driver specified in the system graphics.cfg if no
  % driver is specified.
  % \usepackage[dvips]{graphicx}
  % declare the path(s) where your graphic files are
  % \graphicspath{{../eps/}}
  % and their extensions so you won't have to specify these with
  % every instance of \includegraphics
  % \DeclareGraphicsExtensions{.eps}
\fi
% graphicx was written by David Carlisle and Sebastian Rahtz. It is
% required if you want graphics, photos, etc. graphicx.sty is already
% installed on most LaTeX systems. The latest version and documentation
% can be obtained at: 
% http://www.ctan.org/pkg/graphicx
% Another good source of documentation is "Using Imported Graphics in
% LaTeX2e" by Keith Reckdahl which can be found at:
% http://www.ctan.org/pkg/epslatex
%
% latex, and pdflatex in dvi mode, support graphics in encapsulated
% postscript (.eps) format. pdflatex in pdf mode supports graphics
% in .pdf, .jpeg, .png and .mps (metapost) formats. Users should ensure
% that all non-photo figures use a vector format (.eps, .pdf, .mps) and
% not a bitmapped formats (.jpeg, .png). The IEEE frowns on bitmapped formats
% which can result in "jaggedy"/blurry rendering of lines and letters as
% well as large increases in file sizes.
%
% You can find documentation about the pdfTeX application at:
% http://www.tug.org/applications/pdftex





% *** MATH PACKAGES ***
%
%\usepackage{amsmath}
% A popular package from the American Mathematical Society that provides
% many useful and powerful commands for dealing with mathematics.
%
% Note that the amsmath package sets \interdisplaylinepenalty to 10000
% thus preventing page breaks from occurring within multiline equations. Use:
%\interdisplaylinepenalty=2500
% after loading amsmath to restore such page breaks as IEEEtran.cls normally
% does. amsmath.sty is already installed on most LaTeX systems. The latest
% version and documentation can be obtained at:
% http://www.ctan.org/pkg/amsmath





% *** SPECIALIZED LIST PACKAGES ***
%
%\usepackage{algorithmic}
% algorithmic.sty was written by Peter Williams and Rogerio Brito.
% This package provides an algorithmic environment fo describing algorithms.
% You can use the algorithmic environment in-text or within a figure
% environment to provide for a floating algorithm. Do NOT use the algorithm
% floating environment provided by algorithm.sty (by the same authors) or
% algorithm2e.sty (by Christophe Fiorio) as the IEEE does not use dedicated
% algorithm float types and packages that provide these will not provide
% correct IEEE style captions. The latest version and documentation of
% algorithmic.sty can be obtained at:
% http://www.ctan.org/pkg/algorithms
% Also of interest may be the (relatively newer and more customizable)
% algorithmicx.sty package by Szasz Janos:
% http://www.ctan.org/pkg/algorithmicx




% *** ALIGNMENT PACKAGES ***
%
%\usepackage{array}
% Frank Mittelbach's and David Carlisle's array.sty patches and improves
% the standard LaTeX2e array and tabular environments to provide better
% appearance and additional user controls. As the default LaTeX2e table
% generation code is lacking to the point of almost being broken with
% respect to the quality of the end results, all users are strongly
% advised to use an enhanced (at the very least that provided by array.sty)
% set of table tools. array.sty is already installed on most systems. The
% latest version and documentation can be obtained at:
% http://www.ctan.org/pkg/array


% IEEEtran contains the IEEEeqnarray family of commands that can be used to
% generate multiline equations as well as matrices, tables, etc., of high
% quality.




% *** SUBFIGURE PACKAGES ***
%\ifCLASSOPTIONcompsoc
%  \usepackage[caption=false,font=normalsize,labelfont=sf,textfont=sf]{subfig}
%\else
%  \usepackage[caption=false,font=footnotesize]{subfig}
%\fi
% subfig.sty, written by Steven Douglas Cochran, is the modern replacement
% for subfigure.sty, the latter of which is no longer maintained and is
% incompatible with some LaTeX packages including fixltx2e. However,
% subfig.sty requires and automatically loads Axel Sommerfeldt's caption.sty
% which will override IEEEtran.cls' handling of captions and this will result
% in non-IEEE style figure/table captions. To prevent this problem, be sure
% and invoke subfig.sty's "caption=false" package option (available since
% subfig.sty version 1.3, 2005/06/28) as this is will preserve IEEEtran.cls
% handling of captions.
% Note that the Computer Society format requires a larger sans serif font
% than the serif footnote size font used in traditional IEEE formatting
% and thus the need to invoke different subfig.sty package options depending
% on whether compsoc mode has been enabled.
%
% The latest version and documentation of subfig.sty can be obtained at:
% http://www.ctan.org/pkg/subfig




% *** FLOAT PACKAGES ***
%
%\usepackage{fixltx2e}
% fixltx2e, the successor to the earlier fix2col.sty, was written by
% Frank Mittelbach and David Carlisle. This package corrects a few problems
% in the LaTeX2e kernel, the most notable of which is that in current
% LaTeX2e releases, the ordering of single and double column floats is not
% guaranteed to be preserved. Thus, an unpatched LaTeX2e can allow a
% single column figure to be placed prior to an earlier double column
% figure.
% Be aware that LaTeX2e kernels dated 2015 and later have fixltx2e.sty's
% corrections already built into the system in which case a warning will
% be issued if an attempt is made to load fixltx2e.sty as it is no longer
% needed.
% The latest version and documentation can be found at:
% http://www.ctan.org/pkg/fixltx2e


%\usepackage{stfloats}
% stfloats.sty was written by Sigitas Tolusis. This package gives LaTeX2e
% the ability to do double column floats at the bottom of the page as well
% as the top. (e.g., "\begin{figure*}[!b]" is not normally possible in
% LaTeX2e). It also provides a command:
%\fnbelowfloat
% to enable the placement of footnotes below bottom floats (the standard
% LaTeX2e kernel puts them above bottom floats). This is an invasive package
% which rewrites many portions of the LaTeX2e float routines. It may not work
% with other packages that modify the LaTeX2e float routines. The latest
% version and documentation can be obtained at:
% http://www.ctan.org/pkg/stfloats
% Do not use the stfloats baselinefloat ability as the IEEE does not allow
% \baselineskip to stretch. Authors submitting work to the IEEE should note
% that the IEEE rarely uses double column equations and that authors should try
% to avoid such use. Do not be tempted to use the cuted.sty or midfloat.sty
% packages (also by Sigitas Tolusis) as the IEEE does not format its papers in
% such ways.
% Do not attempt to use stfloats with fixltx2e as they are incompatible.
% Instead, use Morten Hogholm'a dblfloatfix which combines the features
% of both fixltx2e and stfloats:
%
% \usepackage{dblfloatfix}
% The latest version can be found at:
% http://www.ctan.org/pkg/dblfloatfix




% *** PDF, URL AND HYPERLINK PACKAGES ***
%
\usepackage{url}
% url.sty was written by Donald Arseneau. It provides better support for
% handling and breaking URLs. url.sty is already installed on most LaTeX
% systems. The latest version and documentation can be obtained at:
% http://www.ctan.org/pkg/url
% Basically, \url{my_url_here}.




% *** Do not adjust lengths that control margins, column widths, etc. ***
% *** Do not use packages that alter fonts (such as pslatex).         ***
% There should be no need to do such things with IEEEtran.cls V1.6 and later.
% (Unless specifically asked to do so by the journal or conference you plan
% to submit to, of course. )


% correct bad hyphenation here
\hyphenation{op-tical net-works semi-conduc-tor}


\begin{document}
%
% paper title
% Titles are generally capitalized except for words such as a, an, and, as,
% at, but, by, for, in, nor, of, on, or, the, to and up, which are usually
% not capitalized unless they are the first or last word of the title.
% Linebreaks \\ can be used within to get better formatting as desired.
% Do not put math or special symbols in the title.
\title{OPQ Version 2: An Architecture for Distributed, Real-Time, High Performance Power Data Acquisition, Analysis, and Visualization}


\author{\IEEEauthorblockN{Anthony J. Christe, Sergey I. Negrashov, Philip M. Johnson, Dylan Nakahodo, David Badke, David Aghalarpour}
\IEEEauthorblockA{Collaborative Software Development Lab\\
University of Hawaii at Manoa\\
Oahu, Hawaii\\
(achriste, sin8, johnson, dylankhn, davidrb, aghalarp) @ hawaii.edu}}


% make the title area
\maketitle

% As a general rule, do not put math, special symbols or citations
% in the abstract
\begin{abstract}
OpenPowerQuality (OPQ) is a framework that supports end-to-end capture, analysis, and visualizations of distributed real-time power quality (PQ) data. Version 2 of OPQ builds on version 1 by providing higher sampling rates, optional battery backup, end-to-end security, GPS synchronization, pluggable analysis, and a real-time visualization framework. The OPQ project makes real-time distributed power measurements which allows users to see both local PQ events and grid-wide PQ events. The OPQ project is split into three major parts: back-end hardware for making power measurements, middleware for data acquisition and analysis, and a front-end providing visualizations. OPQBox2 is a hardware platform that takes PQ measurements, provides onboard analysis, and securely transfers data to our middleware. The OPQ middleware performs triggering on the OPQBox2 sensor network and performs high-level PQ analysis. The results of our PQ analysis and real-time state of the sensor network are displayed using OPQView. We've collected distributed PQ data from locations across Oahu, Hawaii and have demonstrated our ability to detect both local and grid-wide power quality events.
\end{abstract}

% no keywords




% For peer review papers, you can put extra information on the cover
% page as needed:
% \ifCLASSOPTIONpeerreview
% \begin{center} \bfseries EDICS Category: 3-BBND \end{center}
% \fi
%
% For peerreview papers, this IEEEtran command inserts a page break and
% creates the second title. It will be ignored for other modes.
\IEEEpeerreviewmaketitle



\section{Introduction}
As power grids transition from a centralized distribution model to a distributed model, maintaining a stable grid requires fine grained knowledge of how distributed renewables are affecting the state of the grid. Monitoring PQ on a distributed generation smartgrid requires a consumer level distributed sensor network. How do we collect, analyze, and visualize power quality (PQ) on the power grid scale using data gathered from residential utility customers? 

To answer this question, we developed and deployed an open source hardware and software system focused on consumer level monitoring across Oahu. 

Our system is able to aggregate distributed PQ measurements, request raw PQ data from specific devices, perform high level real-time PQ analysis and classification, and display PQ at both the consumer level and the grid level. At its core is a multi-layered framework consisting of a hardware backend for measuring distributed PQ data, a middleware for triggering and higher level PQ analysis, and a frontend for consumer friendly local and grid level PQ visualizations.

%In late 2014 we performed a pilot study to test the feasibility of our hardware and software system. Data collected during this study showed strong correlations between PV production and daily voltage trends. We demonstrated that a grid wide view allows us to determine if PQ events take place at the consumer level or at the grid level. 

Recently, we've collected [TODO: HOW MUCH] data from [TODO: HOW MANY] devices from devices distributed across Oahu. We found [TODO: WHAT].

Finally, we could produce these boxes for under \$100 each. This is roughly an order of magnitude cheaper than other power quality monitoring equipment and is better suited for our applications.

\section{Related Work}
[TODO:]

\section{OPQ Version 2 Architecture}
\subsection{OPQBox2}
OPQBox2 is a modern, high performance, sensor network centered power quality monitor. OPQBox2 provides high resolution sampled data of up to 50kS/s at 16bits. Optional GPS synchronization and battery backup can be added to the OPQBox2 to tailor it to a specific power quality measurement. 

While the sampling and GPS synchronization is controlled by the realtime DSP, all of the signal processing and communication is controlled by a Raspberry PI single board computer. The Raspberry Pi collects 10 AC cycles of ADC measurements at a time and computes the fundamental frequency and $V_{rms}$. These values are sent to the cloud triggering service, OPQMakai, via WiFi while the raw waveforms are buffered on the OPQBox2 for 1 hour. The waveform data is split into chunks of ten cycles each. Each chunk is stored is a Redis sorted-set, indexed by timestamp. Range queries on raw waveform data occur in $\mathcal{O}(n\log{}n + m)$ time. Older entries in the buffer are expired via a garbage collection process after a configurable amount of time. 

The acquisition cloud service may request raw waveforms from the collection of OPQBox2s if the triggering data stream indicated a power quality event. When the triggering server requests data from an OpqBox2, it requests a range of data. This data is queried in Redis, reserialized into a single large data packet, and transfered back to the requesting server over the same ZeroMQ channel that it arrived on. All of the communication between the OPQBox2 and the cloud services are accomplished via an encrypted ZeroMQ connection.

In this paper we verify the synchronization provided by Network Time Protocol (NTP) by comparing power data collected from two different boxes. We can verify the synchronization by cross-correlating both signals with each other. The resulting signal from the cross correlation is at a maximum equal to the delay of the signals.

We also confirm the frequency resolution of the OPQBox by supplying it a 60 Hz signal from a function generator and observing the data recorded by the box. We plot the frequency data and found that our box can accurately record the frequency supplied by the generator.

\subsection{OPQHub}
OPQHub is OPQ's middleware system and is responsible for requesting fine-grained data from OPQBox2's and for performing high level PQ analysis.  OPQHub is split into two components, OPQMakai, a pluggable triggering and data request component and OPQMauka, a pluggable real-time PQ analysis component.

All communication between the OPQBox2 and the analysis stack passes through two connection brokers. These brokers allow us to terminate encryption at the cloud boundary, thus simplifying the analysis framework. 

ZeroMq is used throughout our cloud infrastructure to connect the analysis services to brokers. 

Triggering stream consists of the feature reduced data(frequency and $V_{rms}$ measurements) for each device in the OPQ2 network. This stream is brokered via the Triggering broker, and analyzed by the set of analysis processes called Makai. If multiple devices show temporally coherent deviation from the norm, Makai will request the raw data from the OPQBox2 devices via the Acquisition broker. 

Further analysis of the raw waveform is performed by OPQMauka analysis and classification system.  Each of the plugins in OPQMauka are responsible for different PQ classification and analysis tasks. We currently implement plugins that detect and report PQ measurements, voltage sags and swells, frequency sags and swells, and perform ITIC event classification of voltage events. We've also developed plugins that, along with the triggering framework OPQMakai, allow us to detect these events on a distributed level.

The publish subscribe nature of the plugin system allows plugins to only subscribe to channels they're interested in and produce results that other plugins may be interested in. This allows us to form a pipeline of processing on the cloud side. For example, the voltage analysis plugin will monitor voltage thresholds and report voltage dips and swells. When the voltage plugin encounters an event, it will produce a voltage event message. The ITIC plugin will consume voltage event messages and classify them by their ITIC classification.

Data products from Mauka are stored in MongoDB and presented to clinets use OPQView.

\subsection{OPQView}
OPQView serves as the front-end user interface of the OpenPowerQuality system. It is implemented via MeteorJS, an open source JavaScript web framework that offers real-time and reactive bi-directional communication between client and server. Unlike other existing full-stack web frameworks, Meteor was developed specifically to offer a reactive front-end experience, making it a fitting choice for the dynamic and data-driven nature of our application. Meteor, although an emerging technology, is built upon Node.js - a widely used JavaScript runtime environment. Meteor’s integration with Node.js grants access to NPM (node package manager) and its large assortment of proven and tested JavaScript modules which we utilize to further enhance the OPQView user experience.

Broadly speaking, the OPQ system consists of three interdependent components - OPQBoxes, OpqHub, and OpqView. OpqView essentially serves the end-point of this system, where power data eventually propagates to for user consumption. As such, OpqView is responsible for displaying collected power quality data in a useful and meaningful manner. OpqView does not receive its data directly from OpqBoxes, but rather through the intermediary OpqHub, which is responsible for collecting and analyzing data from OpqBoxes - classifying power quality events as they occur. OpqView maintains a communications interface with OpqHub through the use of a MongoDB database, from which it is able to reactively retrieve real-time power quality data. This data includes voltages and frequencies of individual devices, as well as classified individual power quality events and their corresponding waveform data. OpqView not only serves as a front-end to display real-time data, but also a platform to safely and securely store a complete history of power quality data for each individual device - allowing for users to further analyze historical PQ event data well after they have occurred on the grid. In addition to local event data individual to each device, users can also view distributed events - collections of detected local events that are likely caused by the same grid-wide power event. Over time, OpqHub can determine patterns in these distributed events, allowing OpqView to display ‘communities’ of devices which have a propensity to experience related power events - leading to further insights on how power events tend to propagate through the local grid. The real-time nature of our data allows for a wide potential of visualization techniques - including real-time heat maps of device voltages and frequencies across the grid. OpqView also supplies the administrative interface for OpqBoxes, allowing device owners to adjust their device’s privacy and sharing settings as needed. Device owners can choose to publicly share their device’s data, or instead only allow a specified set of users access.

Ultimately, we feel that OpqView will serve as a unique platform for power quality analysis and visualization. Meteor’s strong support for reactive data models remains its main draw - and is the primary reason we have chosen it for OpqView’s implementation. The dynamic and data driven nature of our smart-grid application, and the power grid system as a whole, makes it an ideal and exciting choice.

\section{Results}

[TODO:]

% An example of a floating figure using the graphicx package.
% Note that \label must occur AFTER (or within) \caption.
% For figures, \caption should occur after the \includegraphics.
% Note that IEEEtran v1.7 and later has special internal code that
% is designed to preserve the operation of \label within \caption
% even when the captionsoff option is in effect. However, because
% of issues like this, it may be the safest practice to put all your
% \label just after \caption rather than within \caption{}.
%
% Reminder: the "draftcls" or "draftclsnofoot", not "draft", class
% option should be used if it is desired that the figures are to be
% displayed while in draft mode.
%
%\begin{figure}[!t]
%\centering
%\includegraphics[width=2.5in]{myfigure}
% where an .eps filename suffix will be assumed under latex, 
% and a .pdf suffix will be assumed for pdflatex; or what has been declared
% via \DeclareGraphicsExtensions.
%\caption{Simulation results for the network.}
%\label{fig_sim}
%\end{figure}

% Note that the IEEE typically puts floats only at the top, even when this
% results in a large percentage of a column being occupied by floats.


% An example of a double column floating figure using two subfigures.
% (The subfig.sty package must be loaded for this to work.)
% The subfigure \label commands are set within each subfloat command,
% and the \label for the overall figure must come after \caption.
% \hfil is used as a separator to get equal spacing.
% Watch out that the combined width of all the subfigures on a 
% line do not exceed the text width or a line break will occur.
%
%\begin{figure*}[!t]
%\centering
%\subfloat[Case I]{\includegraphics[width=2.5in]{box}%
%\label{fig_first_case}}
%\hfil
%\subfloat[Case II]{\includegraphics[width=2.5in]{box}%
%\label{fig_second_case}}
%\caption{Simulation results for the network.}
%\label{fig_sim}
%\end{figure*}
%
% Note that often IEEE papers with subfigures do not employ subfigure
% captions (using the optional argument to \subfloat[]), but instead will
% reference/describe all of them (a), (b), etc., within the main caption.
% Be aware that for subfig.sty to generate the (a), (b), etc., subfigure
% labels, the optional argument to \subfloat must be present. If a
% subcaption is not desired, just leave its contents blank,
% e.g., \subfloat[].


% An example of a floating table. Note that, for IEEE style tables, the
% \caption command should come BEFORE the table and, given that table
% captions serve much like titles, are usually capitalized except for words
% such as a, an, and, as, at, but, by, for, in, nor, of, on, or, the, to
% and up, which are usually not capitalized unless they are the first or
% last word of the caption. Table text will default to \footnotesize as
% the IEEE normally uses this smaller font for tables.
% The \label must come after \caption as always.
%
%\begin{table}[!t]
%% increase table row spacing, adjust to taste
%\renewcommand{\arraystretch}{1.3}
% if using array.sty, it might be a good idea to tweak the value of
% \extrarowheight as needed to properly center the text within the cells
%\caption{An Example of a Table}
%\label{table_example}
%\centering
%% Some packages, such as MDW tools, offer better commands for making tables
%% than the plain LaTeX2e tabular which is used here.
%\begin{tabular}{|c||c|}
%\hline
%One & Two\\
%\hline
%Three & Four\\
%\hline
%\end{tabular}
%\end{table}


% Note that the IEEE does not put floats in the very first column
% - or typically anywhere on the first page for that matter. Also,
% in-text middle ("here") positioning is typically not used, but it
% is allowed and encouraged for Computer Society conferences (but
% not Computer Society journals). Most IEEE journals/conferences use
% top floats exclusively. 
% Note that, LaTeX2e, unlike IEEE journals/conferences, places
% footnotes above bottom floats. This can be corrected via the
% \fnbelowfloat command of the stfloats package.




\section{Conclusion}
We designed a low cost, real-time, open-source framework capable of aggregating distributed PQ measurements, requesting raw PQ data from specific devices, performing high level real-time PQ analysis and classification, and displaying PQ at both the consumer level and the grid level. 

Our system detected multiple local and grid-level PQ events and can also display the current real-time state of the power grid. [TODO: Finish conclusion]


% trigger a \newpage just before the given reference
% number - used to balance the columns on the last page
% adjust value as needed - may need to be readjusted if
% the document is modified later
%\IEEEtriggeratref{8}
% The "triggered" command can be changed if desired:
%\IEEEtriggercmd{\enlargethispage{-5in}}

% references section

% can use a bibliography generated by BibTeX as a .bbl file
% BibTeX documentation can be easily obtained at:
% http://mirror.ctan.org/biblio/bibtex/contrib/doc/
% The IEEEtran BibTeX style support page is at:
% http://www.michaelshell.org/tex/ieeetran/bibtex/
%\bibliographystyle{IEEEtran}
% argument is your BibTeX string definitions and bibliography database(s)
%\bibliography{IEEEabrv,../bib/paper}
%
% <OR> manually copy in the resultant .bbl file
% set second argument of \begin to the number of references
% (used to reserve space for the reference number labels box)
%\begin{thebibliography}{1}
%
%\bibitem{IEEEhowto:kopka}
%H.~Kopka and P.~W. Daly, \emph{A Guide to \LaTeX}, 3rd~ed.\hskip 1em plus
%  0.5em minus 0.4em\relax Harlow, England: Addison-Wesley, 1999.
%
%\end{thebibliography}




% that's all folks
\end{document}


